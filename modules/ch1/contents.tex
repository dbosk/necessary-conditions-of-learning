\mode*

\section{Disposition}

\begin{frame}
  \begin{figure}
    \includegraphics[height=0.8\textheight]{fig/book.jpg}
    \caption{Front cover of Necessary Conditions of Learning.}
  \end{figure}
\end{frame}

\begin{frame}
  \begin{enumerate}
    \alert{\item What makes humans human?}
    \item What is to be learned?
    \item Sameness and difference in learning
    \item What does the world look like to others?
    \item The art of learning
    \item Making learning possible
    \item Learning to help others to learn
  \end{enumerate}
\end{frame}

\begin{frame}
  \begin{block}{Ch 1 What makes humans human?}
    \begin{itemize}
      \item What is pedagogy?
      \item How does it relate to learning?
    \end{itemize}
  \end{block}
\end{frame}

\section{Pedagogy}

\begin{frame}
  \begin{definition}[Premack's 1984 definition]
    \begin{enumerate}
      \item There is a goal for learning.
      \item There is a systematic attempt (by the \enquote{teacher}) to help 
        the learner to reach that goal.
      \item The systematic attempt is guided by the teacher's perception of the 
        learner's progress (or lack of progress).
    \end{enumerate}
  \end{definition}
\end{frame}

\section{Means-end structure}

\begin{frame}
  \begin{definition}[Teleology]
    \begin{description}
      \item[Simple teleology] what can I use to achieve this goal?
      \item[Inverse teleology] what can I use this object for?
      \item[Recursive teleology] use tools to produce tools to achieve a goal.
    \end{description}
  \end{definition}

  \pause

  \begin{remark}
    \begin{itemize}
      \item For simple and inverse teleology, the relation between means and 
        goal is visible to the learner.
      \item For recursive teleology, the means-end structure is invisible.
    \end{itemize}
  \end{remark}
\end{frame}

\begin{frame}
  \begin{definition}[Emulation and imitation]
    \begin{description}
      \item[Imitation] when the means-end structure is hidden, only imitation 
        is possible.
      \item[Emulation] can discern the critical steps, generalize to other 
        situations.
    \end{description}
  \end{definition}
\end{frame}

\section{Learning as a by-product}

\begin{frame}
  \begin{example}[Learning as a by-product or as an aim]
    \begin{itemize}
      \item Apprenticeship (PhD studies?) is learning as a by-product.
      \item As opposed to schooling, which is learning as an aim.
    \end{itemize}
  \end{example}

  \pause

  \begin{example}[\enquote{De-pedagogized} learning]
    \begin{itemize}
      \item Authentic vs unauthentic learning
      \item \enquote{Real life} vs school
      \item Problem/project-based learning?
    \end{itemize}
  \end{example}
\end{frame}

\begin{frame}
  \begin{example}[The teacher's paradox]
    \blockcquote[p.~13]{NecessaryConditionsOfLearning}{%
      \textins{T}he more the teacher does to enable the students to answer the 
      questions asked, to solve the problems given, the fewer opportunities may 
      be left for the students to learn to understand that which they are 
      expected to learn to understand. The reason is that the more clearly the 
      teacher tells the students what is to be done, the less chance the 
      students get to make the necessary distinctions (for instance, between 
      what is critical and what is not).%
    }
  \end{example}

  \pause

  \begin{remark}[The problem]
    \begin{itemize}
      \item Surface vs deep learning~\cite{DeepSurfaceLearning}.
      \item Study compared US and Chinese teachers in maths.
    \end{itemize}
  \end{remark}
\end{frame}

\section{Conclusion}

\begin{frame}
  \begin{block}{In conclusion}
    \begin{itemize}
      \item This affects how we design our teaching overall.
      \item Are the goals visible to students?
      \item Are our assignments too explicit, too much \enquote{hand-holding}?
      \item Do we teach in a way that yields imitation rather than the desired 
        emulation?
    \end{itemize}
  \end{block}
\end{frame}

\mode<all>{\endinput}

\begin{frame}
  \begin{onlyenv}<1>
    \begin{remark}
      \blockcquote[p.~21]{NecessaryConditionsOfLearning}{%
        Teachers must somehow organize work in the classroom in order to make 
        learning possible.
        A frequent question posed in educational research is:
        Which is the best arrangement for learning?%
      }
    \end{remark}
  \end{onlyenv}
  \begin{example}[{\cite[p.~21]{NecessaryConditionsOfLearning}}]
    \begin{itemize}
      \item Is learning by yourself better than learning by being taught?
      \item Does homework enhance learning?
      \item Is problem-based learning better than lectures for big classes?
      \item Is individualized learning preferable to group work?
      \item Is project work a good idea?
    \end{itemize}
  \end{example}
  \begin{onlyenv}<2>
    \begin{remark}
      \blockcquote[p.~21]{NecessaryConditionsOfLearning}{%
        The problem with questions of this kind is that they cannot be 
        answered.
        It is not that they cannot be answered \emph{yet}, and it is not 
        because of a scarcity of research funds or a scarcity of good ideas.
        They are simply imponderable because of the degree of generality.%
      }
    \end{remark}
  \end{onlyenv}
\end{frame}

\section{Take away}

\begin{frame}
  \begin{center}
    What is to be learned?\\[1em]
    instead of\\[1em]
    What is to be done?
  \end{center}
\end{frame}

\begin{frame}
  \begin{block}{Ch 2 What is to be learned?}
    \begin{itemize}
      \item What is an intended learning outcome?
      \item What should it be?
      \item It depends on the learner: \emph{critical aspects, critical 
        features} --- the atoms of the learning objective.
    \end{itemize}
  \end{block}
\end{frame}
