\documentclass[a4paper,swedish]{article}
\usepackage[utf8]{inputenc}
\usepackage[T1]{fontenc}
\usepackage{csquotes}
\usepackage{babel}
\usepackage{authblk}
\usepackage[style=authoryear-comp]{biblatex}
\addbibresource{bibedu.bib}
\addbibresource{bibliography.bib}
\usepackage{hyperref}
\usepackage{cleveref}

\title{Ansökan om utvecklingsmedel för utveckling av digitalt läromaterial}
\author{Daniel Bosk}
\affil{KTH EECS TCS, \texttt{dbosk@kth.se}}

\begin{document}
\maketitle

\section*{Sammanfattning}

\paragraph{Mål} Målet är att få en start på ett långsiktigt pedagogiskt 
utvecklingsprojekt med titeln \emph{Technology Enhanced Learning Studies}.
Grundidén är att nyttja återanvändbarheten och, således, reproducerbarheten i 
interaktiva, digitala undervisningsverktyg för att bedriva skalbar 
forskningsbaserad digital undervisningsutveckling.
Fokus ligger på digitaliserad undervisning av programmering.

Vinster med projektet?
\begin{itemize}
  \item Det är bra för KTH då det finns ett asynkront, interaktivt, digitalt 
    undervisningsmaterial som är både förankrat i och förmedlar pedagogisk 
    grundforskning~\parencite{NecessaryConditionsOfLearning} samt digitala 
    undervisningsmetoder.
    Detta kan användas för att utbilda handledare, men även utbilda och 
    inspirera lärare till digital undervisning.

  \item Det är bra för KTH då resultaten förbättrar de inledande kurserna i 
    programmering.
    Detta rör främst de kurser jag undervisar själv (ungefär 350 studenter som 
    inte läser datateknik).
    Men resultaten kan med fördel tillämpas även i övriga 
    programmeringsteknikkurser (ungefär 1100 studenter).

  \item Det är bra för samhället då skolor\footnote{%
      Alla grund- och gymnasieskolor ska undervisa i programmering som en del 
      av matematikundervisningen~\parencite{SkolverketGrProg,SkolverketGyProg} 
      från första klass.
      Detta material är relevant från årskurs 7.
      (Jag har tidigare med gott resultat undervisat samma innehåll som i KTH:s 
      introduktion till programmering för årskurs 7.)
    } kommer att bjudas in att använda materialet och genom undervisningen av 
    metoden även kan bidra till den forskningsgrundade utvecklingen.
    Detta kan även gynna utvecklingen av lärarnas undervisning i andra ämnen 
    genom andrahandseffekter.
\end{itemize}

Kostnader med projektet?
\begin{itemize}
  \item Initiala utvecklingskostnader täcks av denna ansökan.
  \item Underhållskostnader för materialet är låg.
    \begin{itemize}
      \item Källkoden till allt material tillgängliggörs under en Creative 
        Commons licens på GitHub, fritt tillgängligt för alla.
      \item Tillgängliggörande av media, såsom video, kommer också vara under 
        en Creative Commons-licens på KTH Play och YouTube.
      \item Interaktiva versioner av mediematerial för plattformen 
        FeedbackFruits\footnote{%
          KTH är på gång att LTI-integrera FeedbackFruits tjänster.
          När det händer kommer det sannolikt vara smidigaste sättet att 
          tillgängliggöra användbart material inom KTH.
          Allt källmaterial som ligger till grund för interaktionen i 
          FeedbackFruits kommer dock att finnas tillgängligt via GitHub.
        } kan göras tillgängligt för användning via min personliga 
        livstidslicens.
    \end{itemize}
\end{itemize}

\paragraph{Utvecklingsmedel} Vi ansöker härmed om 715\,000 SEK för att utveckla 
digitala läromaterial för:
\begin{enumerate}
  \item Utveckla digitalt, återanvändbart introduktionsmaterial till den 
    underliggande teorin och metoden, baserat på 
    \textcite{NecessaryConditionsOfLearning}. Materialet ska användas för att 
    utbilda handledare för att kunna delta.
    (Materialet kan användas senare användas som en kurs i lärande.
    Detaljer i \cref{intrometod}.)

    120 timmar för undertecknad: 120\,000 SEK (LKP 55\%, OH 85\%)

  \item Analysera och revidera (en första omgång) befintligt (digitalt) 
    kursmaterial för DD1310 och DD1315 utifrån den underliggande lärandeteorin 
    (variationsteori, \cite{NecessaryConditionsOfLearning}). (Detta innefattar 
    även OLI-materialet som används i fler kurser i programmeringsteknik, mer 
    om OLI:s relation till teorin i \cref{analys}.)
    
    Baserat på analysen och revideringen behöver vi även utveckla interaktiva 
    videos med frågor för att utvärdera studenternas lärande före och efter de 
    har tagit sig igenom materialet, för att utvärdera materialet. Dessa frågor 
    kan återanvändas för formativ bedömning för studenterna.
    (Detaljer i \cref{analys}.)

    Detta arbete ska avslutas senast i augusti 2021.

    40 timmar för undertecknad: 40\,000 SEK (LKP 55\%, OH 85\%)

    \(3\times 360\) timmar för handledare: 470\,000 SEK (150 SEK/h, 55\% LKP, 
    85\% OH)

  \item Genomföra flertalet iterationer (enligt metoden) av analys av resultat 
    av genomförd undervisning och revidering av material inför ytterligare 
    genomförande under höstens kursomgångar. Vi har två kursomgångar, vilket 
    ger en möjlighet till att revidera och testa igen.
    (Detaljer i \cref{iterationer}.)

    4 h/vecka under P1 för undertecknad: 40\,000 SEK (LKP 55\%, OH 85\%)

    \(3\times 4\) h/v under P1 för handledare: 55\,000 SEK (150 SEK/h, LKP 
    55\%, OH 85\%)
\end{enumerate} 

\section{En introduktion till teori och metod}\label{intrometod}

Vi avser att utveckla ett digitalt, asynkront men interaktivt läromedel.
Innehållsmässigt avser vi att materialet ska behandla teorin (variationsteori, 
fenomenografi) och utvecklingsmetoden (learning studies).
Formatmässigt kommer vi att använda de verktyg för digital undervisning 
(FeedbackFruits) som vi avser att använda senare för materialet i 
programmeringsteknik.
Det blir således ett läromedel för att utveckla digital undervisning väl 
grundat i lärandevetenskaplig teori och metod.

FeedbackFruits tillhandahåller en uppsättning verktyg för digitaliserad 
undervisning som \emph{kompletterar OLI}.
Framförallt ger verktygen tillgång till asynkron interaktion mellan deltagare 
och lärare, något som inte ingår i OLI-modellen.
Exempelvis, interaktiva förinspelade föreläsningar och dokument, där studenter 
kan ställa ställa frågor, starta diskussioner med andra studenter och lärare om 
olika delar av en video eller ett dokument medan de tar sig igenom materialet 
\emph{i egen takt} --- de kan då ta sig tiden att reflektera ordentligt.
Det ger även läraren möjligheten att lägga in quiz på lämpliga ställen för att 
utmana studenterna och testa effektiviteten av 
läromedlet~\parencite{NecessaryConditionsOfLearning}.
Se
\begin{center}
  \url{https://feedbackfruits.com/tool-overview}
\end{center}
för fler exempel.
FeedbackFruits kommer att LTI-integreras under hösten och används redan av 
flera andra lärosäten i Sverige.

Den underliggande teorin och metodiken bygger på
\citetitle{NecessaryConditionsOfLearning}
av 
\citeauthor{NecessaryConditionsOfLearning}~\parencite*{NecessaryConditionsOfLearning}.
Jag har en kort sammanfattande video på
\href{https://eu.feedbackfruits.com/groups/activity-course/d04b0280-e219-42c4-aee1-1272609bc4bd}{FeedbackFruits} 
(bara att registrera ett konto)
eller
\href{https://youtu.be/_d42-kTKDcI}{YouTube}
(samma video, 10 min vid dubbel hastighet, utan interaktion).
Jag rekommenderar att se videon via FeedbackFruits trots att man måste 
registrera ett konto (än så länge). Den visar avsiktligt på några funktioner 
som kommer att användas i det interaktiva videomaterialet. 

Materialet kommer att vara återanvändbart, asynkront, interaktivt och utvecklat 
med learning studies. Materialet kan tillsammans med (summativ) examination 
utgöra en formell kurs i lärandeteorin, forskningsmetoden och 
undervisningsverktygen för andra lärare och handledare.

Materialet behövs för att snabbt skola in handledare i teori och metod för att 
de ska kunna delta i efterföljande arbete med kursmaterialet i 
programmeringsteknik (se \cref{analys}). Jag kommer även att använda materialet 
för att skola in övriga handledare i den underliggande pedagogiken i 
programmeringsteknikkurserna jag ansvarar för.


\section{Analys och revidering av läromedel i programmeringsteknik}%
\label{analys}

Variationsteori~\parencite[kap.~3]{NecessaryConditionsOfLearning} kan användas 
för att analysera undervisningsmaterial och avgöra huruvida det går att lära 
sig någonting överhuvudtaget från ett material under givna förutsättningar. 
Teorin kan således även användas som guide för (om-)design av material.
Även om fokus ligger på materialet som är utanför OLI, så måste OLI ingå i 
analysen för att förstå möjligheterna för lärande i kursmaterialet som helhet. 
Detta gör att analysresultaten även kan föras tillbaka till arbetet med OLI.

Lärandemålen måste brytas ner i kritiska aspekter, eller variationsdimensioner, 
och läromedlet behöver påvisa variation i dessa dimensioner för att 
studenterna ska \emph{kunna} lära sig.
Vi kan hitta dessa kritiska aspekter genom att observera studenter och, 
speciellt, deras missförstånd.
Detta då dessa aspekter är osynliga för dem som redan bemästrar stoffet.
Några av de kritiska aspekterna finns redan dokumenterade i 
forskningslitteraturen~\parencite[exempelvis][]{VariationTheoryInProgramming}, 
men inte andra.

Jag har dessutom en hypotes att dessa variationsdimensioner bör användas som 
det som kallas för \enquote{skills} i OLI; och att användningen av OLI då blir 
ännu effektivare.
Det var dock inte så vi utvecklade de \enquote{skills} som används i 
OLI-materialet vi har idag.
Så detta är något som sannolikt kan föras tillbaka till OLI-materialet.
%(Detta kanske kan vara en del av förklaringen att KTH bara kunde reducera 
%tiden med 25\% istället för CMU:s 50\% i artikeln av 
%\cite{ReducedLearningTimeKTH}.)

\Textcite{NecessaryConditionsOfLearning} hänvisar till litteratur som visar att 
studenterna får ett väsentligen bättre behållande av lärandet när de måste 
fundera innan de får något förklarat.
Därav behöver vi utveckla uppgifter och svar som kan integreras i det digitala 
materialet.
Dessa uppgifter skiljer sig i natur från de stora frågedatabaser som behövs i 
OLI.

Vi behöver även utveckla frågor anpassade till de kritiska aspekterna som dels 
kan användas som formativ bedömning för studenterna, så att de kan se om de 
lärt sig, men även för oss att utvärdera effekten av materialet.

Detta är det huvudsakligt tunga arbetet som behöver genomföras under sommaren.


\section{Iterationer under kursen}\label{iterationer}

Som nämndes ovan behöver vi utvärdera effekten hos materialet samt analysera 
varför och revidera i de fall det inte haft önskad effekt.
För detta behövs tid utöver ordinarie undervisningstid för att kunna analysera 
och revidera inför att nästa klass ska ta del av materialet (bara dagarna 
efter).


\section*{Framtiden och en bredare kontext}

Materialet kommer att tillgängliggöras under en Creative Commons-licens och 
olika intressenter kommer i framtiden att bjudas in att bidra med forskning och 
utveckling.

Ambitionen är att kontinuerligt arbeta enligt teorin och metoden tillsammans 
med handledare som rekryteras från civilingenjör- och lärarprogrammet (CL).
Detta bidrar till deras utbildning, då denna lärandeteori och metod inte 
behandlas i programmet; det bidrar även till deras erfarenhet av 
forskningsbaserad utveckling av undervisning; de får även ett material de kan 
använda sig av i sin undervisning och de kan då fortsätta att bidra forskning 
och utveckling.

Vi avser också att bjuda in skolor i regionen att använda materialet, då de 
behöver undervisa precis det här stoffet från årskurs 7 i grundskolan.
Men skolorna bjuds även in att delta i forskningen och utvecklingen av det.
Vi har redan varit i kontakt med Upplands-Bro kommun, som är intresserade av 
att involvera sina lärare i precis den här typen av verksamhet.

Dessa utvecklingsmedel skulle alltså bidra till att snabbstarta detta projekt 
som jag kallar \emph{Technology Enhanced Learning Studies}\footnote{%
  Finns en två minuter kort video \href{https://youtu.be/uByy1fJ0Yro}{här}.
}.


\printbibliography
\end{document}
