\documentclass[a4paper,british]{article}
\usepackage[utf8]{inputenc}
\usepackage[T1]{fontenc}
\usepackage{csquotes}
\usepackage{babel}
\usepackage{authblk}
\usepackage[style=authoryear-comp]{biblatex}
\addbibresource{bibedu.bib}
\addbibresource{bibliography.bib}
\usepackage{acro}
\DeclareAcronym{TEL}{%
  short = TEL,
  long = Technology-Enhanced Learning,
}
\DeclareAcronym{LS}{%
  short = LS,
  long = Learning Study,
  long-plural-form = Learning Studies,
}
\DeclareAcronym{DBR}{%
  short = DBR,
  long = Design-Based Research,
}
\DeclareAcronym{OLI}{%
  short = OLI,
  long = Open Learning Initiative,
}
\usepackage{hyperref}
\usepackage{cleveref}

\title{Application: Future leaders for strategic educational development}
\author{Daniel Bosk}
\affil{KTH EECS TCS, \texttt{dbosk@kth.se}}

\begin{document}
\maketitle
\tableofcontents

% # Aims
%
% - continued development of already skilled teachers, focus: pedagogical 
%   skills in collegial and scholarly sense.
% - education on all levels
%
% # Goals
%
% - develop their expertise in issues of importance to education of KTH
% - develop their capacity for impactful educational development, also in 
%   collegial and cross-border collaboration, e.g with students, across 
%   disciplines, between universities.
% - develop their ability to develop, use, assess, and communicate pedagogical 
%   knowledge.
%
%
% # Requirements
%
% - applicants have substantial teaching experience
% - completed the courses on teaching and learning in higher education
%
% # Criteria
%
% - Demonstrated pedagogical skills according to self-evaluation (pedagogical 
%   portfolio)
% - Project's usefulness and importance for KTH and scholarly significance
% - Potential for leadership and influence in development of KTH's educational 
%   environment and/or for scholarship in higher education

\section{Self-evaluation}

% 1. Demonstrates skill, experience and creativity in a range of teaching 
%    activities.

\paragraph{Background}

I studied the CL programme at KTH and SU between 2006 and 2011, focus on 
mathematics and computer science for upper secondary and adult education.
So I hold a degree Master of Education as well as a Master of Science in 
Computer Science\footnote{%
  And soon a PhD in Computer Science too.
}
After graduation I started working as lecturer of computer science at Mid 
Sweden University, from 2011 until 2020.
In 2014 I started my PhD in computer science at KTH, focus on security and 
privacy.
I kept my lecturer position in parallel until 2020, when I started as lecturer 
full-time at KTH EECS.

I have taught extensively throughout the years.
I have taught both on campus and online, mostly online in fact, entirely online 
from 2015 onwards.
The students have ranged from computer science majors, to the more 
socio-technical industrial engineering and management students all the way to 
the most maths-aversive students.

\subsection{Experience of different students}

I find the students on the distance programmes the more interesting to teach.
In many cases, those students had worked for many years before deciding to 
change track and started studying.
This is in stark contrast to the campus programmes where only fresh graduates 
from the upper secondary schools can study.

Some students were studying a master's programme on distance.
The last time I gave my information security course for those students, one of 
them was working (in parallel to studies) in a project at Volvo Cars revising 
the security architecture of their cars.

Other students were taking just a two-year university diploma programme on 
distance.
Many had worked for many years, forgotten (or repressed) their upper-secondary 
maths.
Some of them panicked by just hearing the word \enquote{maths} even.
That makes a challenge trying to teach them basic security and giving them an 
idea for how number theory underpins modern cryptography.
Or even calculating the most basic performance metrics in an operating systems 
course.
Many of them expressed \enquote{I wish you would've been my maths teacher in 
school.}

I think that these students are the most interesting to teach.
The closest I get to that at KTH is in my programming courses, where the 
students have never programmed before\footnote{%
  Soon most will have programmed before coming to KTH, since now it's mandatory 
  in primary and secondary school.
}.
Some really struggle, even though they much higher grade averages and 
proficiency for maths.
But it's far from the same, the life and work experience of the distance 
students adds a lot of value.
At KTH I've heard a lot about the student who don't dare to turn on video, ask 
questions or are reluctant to speak during seminars.
That is much less a problem with more adult learners in the student population.
Those adult learners help remove that friction.

\subsection{Teaching experience}

% 2. Designs and develops courses appropriate for the programme and inspiring 
%    for the students.

I have developed a lot of teaching throughout my career.
I'll try to give some highlights.

I used to run a course on operating systems.
I designed it, developed all material.
One student wrote me an email about it, this is an excerpt:
\begin{quote}
  I really enjoyed this course and found it quite fascinating. There was a lot 
  of information to digest in that course book! :) My only complaint was that I 
  think the course book lacked worked examples, when it trying to explain 
  mathematical concepts.

  I particularly enjoyed the lab. I think that's a good way for a student to 
  learn more about a subject, by engaging them in a practical task. It makes it 
  more interesting and thus fun by doing some practical task, rather than just 
  accepting a whole bunch of concepts and acronyms at face value from the 
  course book.
\end{quote}
The lab was a lab assignment I developed.
I wrote a small memory management simulator, so that the students could 
experiment with different page-replacement algorithms.

Since I moved my lecturer position I have taught on the Computer Security 
course.
We have transformed the course to flipped classroom during the pandemic.
I used some videos that I had recorded for my Information Security course at 
Mid Sweden University a few years ago.
This year we noticed that very few students attended the lectures, so we 
surveyed the students to find out why and how they preferred their teaching.
Turned out that the cause was overload, the students had too much to do in 
other courses and could only cope by doing the mandatory things in the security 
course, namely the labs.
But among the survey responses I stumbled over this:
\begin{quote}
  I liked the pre-recorded videos when they were clearly explained with good 
  visualisations on the slides. For me, those are a fine alternative to 
  \enquote{in real life classes}. I feel like Daniel Bosk has delivered the 
    most enjoyable results in this regard.
\end{quote}

I have experimented with a lot of ways to improve the students' learning.
It ranges from the traditional lectures and labs to flipped classroom, 
hackathons and seminar driven modules.

The hackathon labs were full day labs where students had to solve a problem in 
a large group with the supervision of a tutor.
The idea was that the assignment was too difficult for individual students to 
solve on their own or in smaller groups.
We brought a projector to the lab, projected the screen with the code.
One student at a time wrote code (the driver in pair programming) while the 
rest of the class (the navigator) discussed and said what the driver were 
supposed to write.
It was an interesting concept, but required a skilled tutor present who could 
guide the group, ensure active participation from everyone and rotate the 
driver every now and then.

% 3. Demonstrates expertise in the subject matter (CS, security) and particular 
%    the ability to guide students into the subject.

Another construction that I find useful is the seminar driven modules that mix 
seminars with intermittent lab work.
I developed the idea in my Information Security course, for the authentication 
module.
Authentication is a challenging topic to teach, particularly because everyone's 
idea of authentication is username and password.
And those passwords should be at least eight characters, have upper and lower 
case, digits and special characters.
However, research has shown over and over that that's bad practice.
But since that's what the students are bombarded with in everyday life, I must 
must work really hard to break it.
The idea is that they read research on the problem, meet to discuss it and 
design some experiments to perform (e.g.~crack some passwords they thought were 
secure).
Then they perform the experiments before the next seminar where we discuss the 
results and possibly discuss new papers.

% 4. Draws on experience and pedagogical concepts to develop their teaching and 
%    supervision practice, with a focus on enhancing student learning.

Taken over all my teaching, the most influential theories of learning are 
constructivism and sociocultural theory.
We can see traces of them in both the hackathons and the seminar-driven 
teaching above.
The last two years, however, have been dominated by variation theory and 
phenomenography.
All of the theories complement each other.
But what I find very appealing of variation theory is that it pinpoints the 
necessary conditions of achieving a particular learning objective and allows me 
to analyse teaching (material) in detail in this regard.

\subsection{Collegial contributions}

% 5. Contributes to a collegial and collaborative educational culture, to 
%    programme development or thematic development, across the department, 
%    school of KTH.

% 6. Exchanges teaching experiences and ideas with colleagues and/or the wider 
%    higher education community.

I very much enjoy discussing pedagogy and didactics.
I have started a weekly pedagogy/didactic breakfast meeting together with Linda 
Kann in our division.
Otherwise, I try my best to contribute to the Cerise group, where I gave a talk 
on Marton's 
\citetitle{NecessaryConditionsOfLearning}~\citeyear{NecessaryConditionsOfLearning}.
I also contribute to KTH SoTL:
\begin{itemize}
  \item Bosk, Glassey: "When flying blind, bring a co-pilot", KTH SoTL'21.

  \item Bälter, Riese, Bosk, Glassey, Mosavat, Kann: "Question-based learning 
    with digital support in introductory Python programming courses", KTH 
    SoTL'21.
\end{itemize}
The co-pilot paper was later published at ITiCSE'21.

I also try to contribute to events like Storträffen and Öppen nätverksträff 
whenever time permits:
\begin{itemize}
  \item Storträffen (Autumn 2020), chaired a table on teaching online.
  \item Talk on Interaction in Zoom, Nationellt öppen nätverksträff (Spring 
    2021). Announced in pEECS and nationally. Participants from 11 
    universities.
\end{itemize}

I also interact on pedagogy with my former colleagues.
I convinced one of them to start using FeedbackFruits (I had worked on him on 
this since ScalableLearning existed).
Ironically enough, he managed to get LTI-integration at his institution (Mid 
Sweden University) before I got it at KTH (I still haven't gotten it).
We will do some joint work on how to best use FeedbackFruits in teaching to 
bootstrap our colleagues.

Finally, I participate in the Technology Enhanced Learning research group, 
primarily with Olle Bälter, Ric Glassey and Olga Viberg.
The current focus is on question-based learning and using the OLI platform.


\section{Preliminary project idea}

% - Tentative aims

If admitted to the programme, I intend to work on an idea that I've had for a 
while: \emph{Technology Enhanced Learning Studies}.
It's a combination of {\ac{TEL}} and {\ac{LS}}.
The main idea is to use the reusability and, consequently, the reproducibility 
of interactive, digital teaching tools to perform scalable research-based 
digital teaching development using learning studies~\autocite{LearningStudy} as 
the underlying research methodology.


\subsection{Learning studies}

Learning studies~\autocite{LearningStudy} was developed by 
\citeauthor{LearningStudy} and colleagues.
The core concept originates from the Japanese lesson study, but was 
complemented by Marton's variation theory of learning and extended by \ac{DBR}.

Variation theory~\autocite{VariationTheory} allows us to analyze teaching 
(recorded classroom teaching, textbooks, assignments, etc.) to determine if 
learning is possible given the students' assumed prerequisites.
Variation theory can thus also guide development of new materials.

\Ac{DBR}~\autocite{DesignBasedResearch} allows us to iterate through designs of 
teaching, making incremental improvements.
The idea of learning study is to use variation theory for initial design, 
analysis and redesign of each iteration.
Each teaching step requires a pre- and post-test of students' knowledge to 
measure any learning.


\subsection{Technology-enhanced learning}

\Textcite{MeaningsAreAcquired} tested variation theory by reproducing a failed 
learning situation using a teaching tool.
In a classroom teaching situation, the teacher had accidentally varied 
something that was supposed to be invariant according to the theory.
The post-test showed that the students didn't learn as intended by the teacher, 
but rather didn't learn as expected according to the theory.
They wanted to reproduce this situation.
They did that by implementing the classroom situation as a computer program 
that the students interacted with.
In one group, the program acted in line with theory; in the other, the program 
made the teacher's mistake.
Indeed, the theory held true, they found the same results.

There are several interesting points with this example, but we'll only cover 
some of them.
The first point I'd like to make, is that the teacher improvising can be good 
and bad.
In the example above, it's bad, but there are other examples where it's 
good~\autocite[cf.][for good examples]{NecessaryConditionsOfLearning}.
Interactive teaching tools (or \ac{TEL} tools), such as Online Learning 
Initiative~\autocite{OLIstatistics} and FeedbackFruits\footnote{%
  Both of which we have access to at KTH.
}, can be used to do such programmed interactive teaching as 
\citeauthor{MeaningsAreAcquired} did above.

My second point is that we need to measure learning to see what parts of our 
teaching has effect.
The students must pass the exam, so they will make sure that they do that, with 
or without our help.
They will discuss with each other, find complementary sources, find better 
sources altogether, and use those for studying instead of our teaching.
We can't measure this with any course evaluation, that's too blunt a tool.
We can't use the course assessment either, we don't know if the students 
studied with only the course material or if they get taught by a third party 
(YouTube, StackExchange, etc.).

My last point is about \ac{TEL} tools in general.
\Ac{TEL} tools can easily and suitably replace the traditional lecture.
Few students dare to ask questions during lectures, many students are lost and 
confused before we reach the end of the lecture.
Many students have pointed out the advantage of the lectures during the 
pandemic as getting the video recording, so that they can watch it at their own 
pace, pause or rewind as needed.
Tools like FeedbackFruits allows the students to ask questions, discuss content 
during the videos --- asynchronously with the teacher or their peers.
FeedbackFruits can additionally integrate quizzes into a video, so that 
students can test their understanding and the teacher can see how the students 
are doing.


\subsection{Technology-enhanced learning-studies}

% - Intended results

\paragraph{Intention}

The idea of Technology-Enhanced Learning Studies is to systematically work with 
\ac{TEL} tools and learning studies to continuously research the teaching in 
our courses while we give them.
For instance, we can integrate pre- and post-test of an interactive video into 
the video itself, decreasing the risk of the students getting taught by a third 
party or that there's a significant delay between the teaching and the tests.
If we do this with other parts too, we can track where most learning happens 
and find inefficiencies in our teaching.
However, there is some work to do.

% - What knowledge is needed

\paragraph{What to investigate}

For starters, what should the pre-test look like?
For instance, if the students \emph{learn} something from the pre-test, what 
would that mean for the evaluation of the teaching?
Or should the pre-test be considered part of the teaching?
Probably, it would make sense since that type of construction increases 
learning and retention~\autocite{Szekely1950,BransfordSchwartz1999}.
But then, what does the pre-test say about the learning material?
How do we know if not only the pre-test was sufficient for the students' 
learning?
But maybe that doesn't matter?

Another question, less philosophical and more geared towards practice: how can 
we develop a framework to easily integrate learning studies into our teaching 
using the teaching tools available (Canvas, FeedbackFruits, OLI)?

A related, and perhaps more interesting, question: how can we systematically 
record the students' mistakes, or rather misconceptions, to learn from them?
The core of variation theory is that when teaching we must introduce variation 
in one aspect of a learning objective at a time.
The theory postulates\footnote{%
  But with significant empirical support!
} that each misconception correspond to not having discerned one or more 
aspects of the learning objective.
These should thus be identified and guide systematic development of the 
teaching material.

% - Plans for the work
% - Collaborators

%\paragraph{Collaborators}
%
%I already do work on OLI together with Ric Glassey (EECS TCS) and Olle Bälter 
%(EECS MID).
%That work would also benefit from this.
%
%I also do work with former colleagues of mine at Mid Sweden University on using 
%FeedbackFruits to improve teaching and I plan to try this method with them.
%
%I would like to expand my collaborations.
%There are two groups that are interesting:
%Ference Marton's and Angelika Kullberg's groups at Gothenburg University and
%UpCERG at Uppsala University.
%Variation theory and learning study originates from Marton's group.
%Kullberg has worked a lot with learning studies in mathematics.
%UpCERG (e.g.~Anna Eckerdal) has done some research using variation theory in 
%the area of computer science, but the results are limited.
%Neither of them has had the strong technological aspect that I propose.


\subsection{How can this project contribute to KTH?}

% ## Motivation
%
% - How can the work contribute to KTH?
% - Why is this important, what needs are addressed?
% - What can my work contribute to the education programmes and the educational 
%   environment?
% - How does this project connect to the desired development at KTH and the 
%   school? According to
%     - current visions,
%     - development plans,
%     - action plans?
%
% - Life-long learning
% - Student preferences

Many things suggest that KTH must transition its teaching post-covid.
Partly visions, e.g.~more digital teaching.
But also the students like to have video complement, or even replace, the 
traditional lectures.
What I find the most interesting though, is the life-long learning perspective.
There are would-be students who simply cannot study if the only option is the 
traditional campus studies\footnote{%
  I've had many such students while teaching distance courses given to distance 
  study programmes.
  I've had one such student at KTH, he started because he knew that with the 
  pandemic, KTH would be forced to do distance education.
}.

This project will provide a framework for working with education technologies 
that digitalize teaching, that we have access too and use them to achieve 
teaching based on research practice.
The results should be made available to other teachers to apply in their 
courses.

%Why would we need to always to these pre- and post-tests, even when teaching is 
%working?
%Because prerequisites changes:
%When the teacher changes in a prerequisite course, s/he might no longer cover 
%that exact example that the students needed to see to understand our course.
%When we ourselves updated the preceding module in the same course, we might 
%have changed a something so that a critical aspect is no longer provided to the 
%students.
%When we gave that lecture, we didn't have time to cover the last example in the 
%slides, thus preventing the students from 
%learning~\autocite[cf.][above]{MeaningsAreAcquired}.


\printbibliography
\end{document}
