\mode*

\section{Technology Enhanced Learning Studies}

\begin{frame}
  \begin{block}<2>{Technology Enhanced Learning}
    \hitem E-learning
    \qquad
    \hitem Asynchronicity
    \qquad
    \hitem Reproducibility
  \end{block}

  \begin{center}
    \Large
    \onslide<1,2,4>{Technology Enhanced}
    \onslide<1-4>{Learning}
    \onslide<1,3,4>{Studies}
  \end{center}

  \begin{block}<3>{Learning Studies}
    \hitem Variation theory~\cite{VariationTheory}
    \qquad
    \hitem Design-based research~\cite{DesignBasedResearch}
  \end{block}
\end{frame}

\section{Application and impact}

\begin{frame}
  \begin{block}{Systematizing this}
    \begin{itemize}
      \item<+> Apply to intro programming courses.
        \begin{itemize}
          \item Many students (300+/year).
          \item Different \enquote{types} of students (three programmes).
        \end{itemize}
      \item<+> Recruit TAs from CL programme.
        \begin{itemize}
          \item Well-versed in theories of learning.
          \item Can contribute to research.
          \item They learn and practice methodology.
          \item They get material to use (CC-BY-SA).
        \end{itemize}
      \item<+> Open teaching to schools.
        \begin{itemize}
          \item Teaching has a scalable design (TEL/online).
          \item Must teach programming in maths.
          \item CL TAs move to schools when graduating.
          \item Schools can contribute data to and participate in research.
        \end{itemize}
    \end{itemize}
  \end{block}
\end{frame}
