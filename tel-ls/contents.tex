\mode*

\section{Technology Enhanced Learning Studies}

\begin{frame}
  \begin{block}<2,4>{Technology Enhanced Learning}
    \hitem Asynchronous + Interactive
    \quad
    \hitem Reproducible + Reusable
  \end{block}

  \begin{center}
    \Large\bfseries
    \onslide<1,2,4>{Technology Enhanced}
    \onslide<1-4>{Learning}
    \onslide<1,3,4>{Studies}
  \end{center}

  \begin{block}<3,4>{Learning Studies}
    \hitem Variation theory~\cite{VariationTheory}
    \qquad
    \hitem Design-based research~\cite{DesignBasedResearch}
  \end{block}

  \begin{onlyenv}<5>
    \begin{alertblock}{Output}
      \begin{itemize}
        \item Teaching material for intro programming,
        \item well-founded in pedagogic/didactic research.
      \end{itemize}
    \end{alertblock}
  \end{onlyenv}
\end{frame}

\section{Application and impact}

\begin{frame}
  \begin{block}<+>{Impact for KTH}
    \begin{itemize}
      \item Apply to teaching in intro programming courses.
        \begin{itemize}
          \item Many students (300+/year).
          \item Different \enquote{types} of students (three programmes).
        \end{itemize}
      \item Recruit TAs from CL programme.
        \begin{itemize}
          \item Well-versed in other theories of learning.
          \item They learn and practice this methodology.
          \item They get material to use (CC-BY-SA) in future.
        \end{itemize}
    \end{itemize}
  \end{block}

  \begin{block}<+>{Impact in society: open teaching for schools}
    \begin{itemize}
      \item Teaching is scalable by design (TEL/online).
      \item Must teach programming in maths.
      \item CL TAs move to schools when graduating.
      \item Schools can contribute data to and participate in research.
    \end{itemize}
  \end{block}
\end{frame}
