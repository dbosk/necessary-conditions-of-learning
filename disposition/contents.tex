\mode*

\begin{frame}
  \tableofcontents
\end{frame}

\section{Disposition}

\begin{frame}
  \begin{enumerate}
    \item What makes humans human?
      \hfill
      (forms of learning)
    \item What is to be learned?
      \hfill
      (learning objectives)
    \item Sameness and difference in learning
      \hfill
      (variation theory)
    \item What does the world look like to others?
      \hfill
      (phenomenography)
    \item The art of learning
      \hfill
      (learning alone, doing research)
    \item Making learning possible
      \hfill
      (teaching)
    \item Learning to help others to learn
      \hfill
      (education research)
  \end{enumerate}
\end{frame}


\section{Theory}

\begin{frame}
  \begin{enumerate}
    \alert{
    \item What makes humans human?
      \hfill
      (forms of learning)
    }
    \item What is to be learned?
      \hfill
      (learning objectives)
    \item Sameness and difference in learning
      \hfill
      (variation theory)
    \item What does the world look like to others?
      \hfill
      (phenomenography)
    \item The art of learning
      \hfill
      (learning alone, doing research)
    \item Making learning possible
      \hfill
      (teaching)
    \item Learning to help others to learn
      \hfill
      (education research)
  \end{enumerate}
\end{frame}

\begin{frame}
  \begin{block}{Ch 1 What makes humans human?}
    \begin{itemize}
      \item What is pedagogy?
      \item How does it relate to learning?
    \end{itemize}
  \end{block}

  \begin{example}<2>[Learning as a by-product or as an aim, 
    {\cite[pp.~9--10]{NecessaryConditionsOfLearning}}]
    \begin{itemize}
      \item Apprenticeship (PhD studies?) is learning as a by-product.
      \item Schooling is learning as an aim.
    \end{itemize}
  \end{example}

  \begin{example}<3>[\enquote{De-pedagogized} learning, 
    {\cite[pp.~11-12]{NecessaryConditionsOfLearning}}]
    \begin{itemize}
      \item Authentic vs unauthentic learning
      \item \enquote{Real life} vs school
    \end{itemize}
  \end{example}
\end{frame}

\begin{frame}
  \begin{example}[The teacher's paradox]
    \blockcquote[p.~13]{NecessaryConditionsOfLearning}{%
      \textins{T}he more clearly the teacher tells the students what is to be 
      done, the less chance the students get to make the necessary distinctions 
      (for instance, between what is critical and what is not).%
    }
  \end{example}
\end{frame}

\begin{frame}
  \begin{center}
    What is to be learned?\\[1em]
    instead of\\[1em]
    What is to be done?
  \end{center}
\end{frame}

\begin{frame}
  \begin{enumerate}
    \item What makes humans human?
      \hfill
      (forms of learning)
    \alert{
    \item What is to be learned?
      \hfill
      (learning objectives)
    }
    \item Sameness and difference in learning
      \hfill
      (variation theory)
    \item What does the world look like to others?
      \hfill
      (phenomenography)
    \item The art of learning
      \hfill
      (learning alone, doing research)
    \item Making learning possible
      \hfill
      (teaching)
    \item Learning to help others to learn
      \hfill
      (education research)
  \end{enumerate}
\end{frame}

\begin{frame}
  \begin{block}{Ch 2 What is to be learned?}
    \begin{itemize}
      \item What is an intended learning outcome?
      \item What should it be?
      \item It depends on the learner: \emph{critical aspects, critical 
        features} --- the atoms of the learning objective.
    \end{itemize}
  \end{block}

  \pause

  \begin{block}{The goal}
    \begin{itemize}
      \item \textcquote{NecessaryConditionsOfLearning}{The focus of the theory 
        elaborated in this book is on learning to handle new situations in 
      powerful ways.}
    \end{itemize}
  \end{block}
\end{frame}

\begin{frame}
  \begin{figure}
    \begin{subfigure}{0.3\columnwidth}
      \centering
      \includegraphics{fig/contrast-color.tikz}
      \caption{Contrast}
    \end{subfigure}
    \hfill
    \begin{subfigure}{0.3\columnwidth}
      \centering
      \includegraphics{fig/generalization-color.tikz}
      \caption{Generalization}
    \end{subfigure}
    \hfill
    \begin{subfigure}{0.3\columnwidth}
      \centering
      \includegraphics{fig/fusion-color.tikz}
      \caption{Fusion}
    \end{subfigure}
    \caption{%
      Illustrating the patterns of variation for aspects color and shape.
    }
  \end{figure}

  \begin{onlyenv}<1>
    \begin{example}
      \begin{itemize}
        \item Critical aspect: colour.
        \item Critical feature: blue.
        \item Non-critical feature: green.
        \item Non-critical aspect: shape.
        \item Non-critical features: circle, square.
      \end{itemize}
    \end{example}
  \end{onlyenv}
  \begin{onlyenv}<2>
    \begin{remark}
      \begin{itemize}
        \item There are other aspects too, \emph{unintentionally}.
        \item Another aspect in this figure is order: the circle is always 
          \enquote{first} (to the right).
      \end{itemize}
    \end{remark}
  \end{onlyenv}
\end{frame}

\begin{frame}
  \begin{figure}
    \includegraphics[height=0.5\textheight]{fig/maja-writing.jpg}
    \caption{Child's writing.
      (Corresponds to~\cite[Fig.~2.1, p.~30]{NecessaryConditionsOfLearning}.)
    }
  \end{figure}
  \begin{example}
    \begin{itemize}
      \item Obviously knows some aspect(s) of writing, just not all.
      \item Even the camera app detected this as text.
    \end{itemize}
  \end{example}
\end{frame}

\begin{frame}
  \begin{enumerate}
    \item What makes humans human?
      \hfill
      (forms of learning)
    \item What is to be learned?
      \hfill
      (learning objectives)
    \alert{
    \item Sameness and difference in learning
      \hfill
      (variation theory)
    }
    \item What does the world look like to others?
      \hfill
      (phenomenography)
    \item The art of learning
      \hfill
      (learning alone, doing research)
    \item Making learning possible
      \hfill
      (teaching)
    \item Learning to help others to learn
      \hfill
      (education research)
  \end{enumerate}
\end{frame}

\begin{frame}
  \begin{block}{Ch 3 Sameness and difference in learning}
    \begin{itemize}
      \item The problem with direct reference (induction).
      \item The patterns: contrast, generalization, fusion.
      \item \textcquote[p.~71]{NecessaryConditionsOfLearning}{%
          \textins{P}racticing something other than what was tested was more 
          effective than practicing exactly what was tested.%
        }
    \end{itemize}
  \end{block}

  \begin{example}<2>[Using known to prepare for unknown, 
    {\cite[pp.~67--71]{NecessaryConditionsOfLearning}}]
    \begin{itemize}
      \item Kids were to hit a target by throwing a shuttlecock (badminton 
        \enquote{ball}).
    \end{itemize}
    \begin{enumerate}
      \item Practicing from the same position as testing.
      \item Practicing from several positions, \emph{except the testing 
        position}.
    \end{enumerate}
  \end{example}
\end{frame}

\begin{frame}
  \begin{example}[Physics, {\cite[p.~58]{NecessaryConditionsOfLearning}}]
    \begin{enumerate}
      \item Students were introduced to modern physics.
      \item Students were introduced to modern physics, contrasted with what we 
        believed before.
    \end{enumerate}
  \end{example}

  \pause

  \begin{remark}
    \begin{itemize}
      \item Makes one think about history of maths.
      \item Common to have as a separate course.
      \item CS history?
    \end{itemize}
  \end{remark}
\end{frame}

\begin{frame}
  \begin{enumerate}
    \item What makes humans human?
      \hfill
      (forms of learning)
    \item What is to be learned?
      \hfill
      (learning objectives)
    \item Sameness and difference in learning
      \hfill
      (variation theory)
    \alert{
    \item What does the world look like to others?
      \hfill
      (phenomenography)
    }
    \item The art of learning
      \hfill
      (learning alone, doing research)
    \item Making learning possible
      \hfill
      (teaching)
    \item Learning to help others to learn
      \hfill
      (education research)
  \end{enumerate}
\end{frame}

\begin{frame}
  \begin{block}{Ch 4 What does the world look like to others?}
    \begin{itemize}
      \item This chapter is about phenomenography.
      \item It answers the question \enquote{how can we find out what things 
        look like to other people?}
      \item Also tells us something about assessment\footnote{%
          Marton contrasts Bloom and SOLO taxonomies and phenomenography.
        } and how to find those critical aspects.
    \end{itemize}
  \end{block}

  \pause

  \begin{example}
    \begin{itemize}
      \item \textcquote[pp.~112--113]{NecessaryConditionsOfLearning}{%
          \textins{U}nderstanding does not cause acts; instead, acts express 
          understanding.%
        }
    \end{itemize}
  \end{example}
\end{frame}

\begin{frame}
  \begin{remark}[Philosophical/theoretical view]
    \begin{itemize}
      \item No one sees the world \enquote{as it is}.
      \item The world is viewed from one's own \emph{perspective}, always 
        through the lens of the self.
      \item \emph{The learner should learn to view things in more powerful 
        ways.}
    \end{itemize}
  \end{remark}

  \pause

  \begin{remark}
    \begin{itemize}
      \item Another \emph{view} of this is \enquote{mental models}.
    \end{itemize}
  \end{remark}
\end{frame}

\begin{frame}
  \begin{remark}[Assessment]
      \textcquote[p.~89]{NecessaryConditionsOfLearning}{%
          If we want to find out to what extent they have learned to do so, we 
          should not point out those aspects for them but let the students 
          discern them by themselves.
          Only then might we be able to find out how they deal with novel 
          situations, how they handle the unknowns by the means of the known.%
        }
  \end{remark}
\end{frame}

\begin{frame}
  \begin{example}[Newton's first law, nationellt prov in physics, year 11]
    \textcquote[p.~90]{NecessaryConditionsOfLearning}{%
      In an experiment with a ball, it is found that when the ball falls, it is 
      affected by the air braking force \(F\), which is proportional to the 
      velocity of the ball, \(v\), that is \(F = kv\) where \(k\) in this case 
      is \SI{0.32}{\newton\second\per\metre}. What would the final velocity of 
      the ball be if it were dropped from a high altitude?
      The ball's mass is \SI{0.20}{\kilogram}.%
    }
  \end{example}

  \pause

  \begin{remark}
    \begin{itemize}
      \item \textcquote[p.~90]{NecessaryConditionsOfLearning}{%
          The learners do not have to discern what aspects of the event have to 
          be taken into consideration: they all are pointed out in the 
          problem.%
        }
      \item They must know gravitational force, \(mg\).
      \item And that eventually the forces will balance out.
    \end{itemize}
  \end{remark}
\end{frame}

\begin{frame}
  \begin{onlyenv}<1-2>
    \begin{remark}[{\cite[p.~1]{RigorousMathematicalThinking}}]
      \begin{itemize}
        \item \enquote{There are 26 sheep and 10 goats on a ship.
          How old is the ship's captain?}
        \item This and similar questions given to primary school students in a 
          number of European countries.
        \item \enquote{\emph{More than 60\% of students attempted to solve the 
            problem by combining the given numbers}} \parentext{citing 
          \cite{Verschaffel1999}, my emphasis}.
      \end{itemize}
    \end{remark}
  \end{onlyenv}

  \begin{example}<2-3>[Newton's first law; Johansson, Marton, Svensson 1985]
    \textcquote[p.~91]{NecessaryConditionsOfLearning}{%
      A car is driven at a high constant speed on a motorway.
      What forces act on the car?
    }
  \end{example}

  \begin{onlyenv}<3>
    \begin{remark}
      \begin{itemize}
        \item Relates back to the \enquote{Teacher's Paradox} (Ch 1).
        \item \textcquote[p.~13]{NecessaryConditionsOfLearning}{%
            \textins{T}he more clearly the teacher tells the students what is 
            to be done, the less chance the students get to make the necessary 
            distinctions (for instance, between what is critical and what is 
            not).%
          }
      \end{itemize}
    \end{remark}
  \end{onlyenv}
\end{frame}


\section{Theory in practice}

\begin{frame}
  \begin{enumerate}
    \item What makes humans human?
      \hfill
      (forms of learning)
    \item What is to be learned?
      \hfill
      (learning objectives)
    \item Sameness and difference in learning
      \hfill
      (variation theory)
    \item What does the world look like to others?
      \hfill
      (phenomenography)
    \alert{
    \item The art of learning
      \hfill
      (learning alone, doing research)
    }
    \item Making learning possible
      \hfill
      (teaching)
    \item Learning to help others to learn
      \hfill
      (education research)
  \end{enumerate}
\end{frame}

\begin{frame}
  \begin{block}{Ch 5 The art of learning}
    \begin{itemize}
      \item How do people do to learn new things?
      \item Deep and surface learning
      \item Scientific discoveries as learning
    \end{itemize}
  \end{block}

  \pause

  \begin{example}[{\cite[p.~129]{NecessaryConditionsOfLearning}}]
    \begin{itemize}
      \item Children naturally generate the necessary patterns of variation and 
        invariance.
    \end{itemize}
  \end{example}

  \pause

  \begin{question}
    \begin{itemize}
      \item Do we kill that ability in the (Western) school system?
    \end{itemize}
  \end{question}
\end{frame}

\begin{frame}
  \begin{remark}
    \textcquote[p.~147]{NecessaryConditionsOfLearning}{%
      How we learn and what we learn are fundamentally intertwined.%
    }
  \end{remark}

  \begin{example}[Deep and surface learning, 
    {\cite[pp.~142--147]{NecessaryConditionsOfLearning}}]
    \begin{description}
      \item[Surface] try to memorize.
      \item[Deep] generate patterns of variation and invariance for oneself.
    \end{description}
  \end{example}
\end{frame}

\begin{frame}
  \begin{example}<+>[Copernicus and science, 
    {\cite[pp.~155--156]{NecessaryConditionsOfLearning}}]
    \begin{itemize}
      \item Science: collective learning.
      \item Copernicus' contribution was not the heliocentric view, it was 
        \emph{centricity}.
      \item He opened the dimension of centricity, where heliocentric is a 
        feature different from geocentric.
      \item But that allowed us to consider centricity of many other systems.
    \end{itemize}
  \end{example}

  \begin{example}<+>[Darwin and evolution, 
    {\cite[pp.~157--160]{NecessaryConditionsOfLearning}}]
    \begin{itemize}
      \item Darwin also opened a dimension: species adapt, not replaced.
      \item Reproduced by Wallace doing the same thing independently.
    \end{itemize}
  \end{example}
\end{frame}

\begin{frame}
  \begin{enumerate}
    \item What makes humans human?
      \hfill
      (forms of learning)
    \item What is to be learned?
      \hfill
      (learning objectives)
    \item Sameness and difference in learning
      \hfill
      (variation theory)
    \item What does the world look like to others?
      \hfill
      (phenomenography)
    \item The art of learning
      \hfill
      (learning alone, doing research)
    \alert{
    \item Making learning possible
      \hfill
      (teaching)
    }
    \item Learning to help others to learn
      \hfill
      (education research)
  \end{enumerate}
\end{frame}

\begin{frame}
  \begin{block}{Ch 6 Making learning possible}
    \begin{itemize}
      \item How can we use this theory for teaching?
      \item Covers differences between cultures.
      \item Covers many studies on teaching.
    \end{itemize}
  \end{block}
\end{frame}

\begin{frame}
  \begin{enumerate}
    \item What makes humans human?
      \hfill
      (forms of learning)
    \item What is to be learned?
      \hfill
      (learning objectives)
    \item Sameness and difference in learning
      \hfill
      (variation theory)
    \item What does the world look like to others?
      \hfill
      (phenomenography)
    \item The art of learning
      \hfill
      (learning alone, doing research)
    \item Making learning possible
      \hfill
      (teaching)
    \alert{
    \item Learning to help others to learn
      \hfill
      (education research)
    }
  \end{enumerate}
\end{frame}

\begin{frame}
  \begin{block}{Ch 7 Learning to help others to learn}
  \end{block}
\end{frame}

\section{Summary}

\begin{frame}
  \begin{block}{Take aways for practice}
    \begin{itemize}
      \item Variation theory
      \item Phenomenography
      \item Learning studies
    \end{itemize}
  \end{block}

  \begin{block}{Beautiful foundation}
    \begin{itemize}
      \item Explains other learning theories, \eg Vygotsky~\cite{Vygotsky} and 
        Montessori.
      \item Takes other papers' results, explain them with the theory.
    \end{itemize}
  \end{block}
\end{frame}

\begin{frame}
  \begin{alertblock}{Conjecture}
    \begin{itemize}
      \item OLI works due to the quiz-question feedback yields \emph{contrast}.
      \item Material usually focus on induction (\emph{generalization} without 
        contrast).
      \item \Ie students get the necessary pattern of variation.
    \end{itemize}
  \end{alertblock}
\end{frame}

\begin{frame}
  \begin{alertblock}{Conjecture}
    \begin{itemize}
      \item Students need to repeat things 8 times to learn.
      \item That's how many times they need to discern the critical 
        aspects/features.
      \item With properly designed material\footnote{%
          With the possibility to experience the necessary patterns of 
          variations.
        }, this number can be reduced.
    \end{itemize}
  \end{alertblock}
\end{frame}

\begin{frame}
  \begin{block}{Ending quote}
    \textcquote[p.~246, my emphasis]{NecessaryConditionsOfLearning}{%
          \enquote{Top scores from Shanghai stun educators,} wrote the 
          \emph{New York Times} on December 7, 2010.
          \textelp{}
          A number of different explanations were presented in the article 
          \textins{by journalists, politicians, education experts}.
          Interestingly, there was not a single word said about the things that 
          this whole book is dealing with: how the different subjects are 
          learned and taught. \emph{Could it not be the case that students are 
            better at solving quadratic equations because they are taught to 
          solve quadratic equations in more powerful ways?}%
        }
  \end{block}
\end{frame}

